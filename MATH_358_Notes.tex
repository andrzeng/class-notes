\documentclass[12pt]{article}
\usepackage[utf8]{inputenc}
\usepackage{graphicx} % Allows you to insert figures
\usepackage{amsmath} % Allows you to do equations
\usepackage{fancyhdr} % Formats the header
\usepackage{geometry} % Formats the paper size, orientation, and 
\usepackage{amsthm, amssymb}
\usepackage{textcomp}
\usepackage{tcolorbox}

\begin{document}
\section{Lecture 1: January 6, 2023}
\begin{tcolorbox}
An alternative formulation of the triangle inequality:\\
$\|x - y\| \geq \|x\| - \|y \|$
\end{tcolorbox}
\section{Lecture 2: January 10, 2023}
In 1D, the derivative of a function f at $x_0 \in \mathbb{R}$ is defined as \\
$f'(x_0) = \lim_{x\rightarrow x_0} \dfrac{f(x) - f(x_0)}{x - x_0}$\\
If this limit exists, we say f is \textbf{differentiable} at $x_0$.\\
To prepare our transition into multiple dimensions, we can rewrite the above equation as:\\
$\lim_{x \rightarrow x_0} \dfrac{f(x) - f(x_0)}{x - x_0} - f'(x_0) = 0$\\
Or,\\
$\lim_{x \rightarrow x_0} \dfrac{f(x) - f(x_0) - f'(x_0)(x - x_0)}{x - x_0} = 0$\\
This is the same as questioning the existence of an $L = f'(x_0)$ that satisfies:\\
$\lim_{x \rightarrow x_0} \dfrac{f(x) - f(x_0) - L(x - x_0)}{x - x_0} = 0$\\



\begin{comment}
\begin{tcolorbox}
A function $f: \Omega \subseteq \mathbb{R}^n \rightarrow \mathbb{R}^m$ \textbf{is differentiable at $x_0 \in \mathbb{R}^n$} \\
if $\forall \epsilon > 0, \exists \delta > 0$ such that there exists a linear map $L: \mathbb{R}^n \rightarrow \mathbb{R}^m$ such that when $\|h\|_{\mathbb{R}^n} < \delta$, we have $\|f(x_0 + h) - f(x_0) - L \cdot h\|_{\mathbb{R}^m} < \epsilon \|h\|_{\mathbb{R}^n}$.
\end{tcolorbox}
\end{comment}


\end{document}


